\section{Poisson distribution}

For this exercise, I was asked to write a function that returns the Poisson probability distribution 
$P_\lambda(k)=\frac{\lambda^ke^{-\lambda}}{k!}$ for integers $k$ given a positive mean $\lambda$.

It seems there are three potential problems to tackle. Firstly, there is the possibility of underflow for large values of $k$,
caused by the division by a very large number $k!$. Secondly, there is the factor $\lambda^k$, which can cause overflow for large values of $\lambda$ and/or $k$.
Thirdly, there is the factor $e^{-\lambda}$, which can become too small for large values of $\lambda$.

I implemented three functions, which all compute the Poisson probability distribution in a different way.
The first function only uses a conversion to log space, such that
\begin{align*}
    P_\lambda(k)&=\exp(\ln(\frac{\lambda^ke^{-\lambda}}{k!}))\\
    &=\exp(\ln(\lambda^k)+\ln(e^{-\lambda})-\ln(k!))\\
    &=\exp(k\ln(\lambda)-\lambda-\sum_{i=0}^{k-1}\ln(k-i)).
\end{align*}
The second function first uses not log space but a different order of operations, given by
\begin{align*}
    P_\lambda(k)&=e^{-\lambda}\cdot\frac{\lambda}{k}\cdot\frac{\lambda}{k-1}\cdot\dots\cdot\frac{\lambda}{1}.
\end{align*}
The third function uses a combination of this different order of operations and log space, such that
\begin{align*}
    P_\lambda(k)&=\exp(\ln(\frac{\lambda}{k}\cdot\frac{\lambda}{k-1}\cdot\dots\cdot\frac{\lambda}{1})-\lambda).
\end{align*}

The script in which these functions are implemented is:
\lstinputlisting{1_poisson_distribution.py}

The results are given by the following. The first column states the values for $\lambda$ and $k$, 
the columns named $P1, P2, P3$ give the results for the first, second and third function respectively.
\lstinputlisting{poisson_distribution.txt}

We see that the results of the three functions are the same up to 6 significant digits for the first three values of $\lambda,k$.
This is to be expected, as the corresponding values of $\lambda$ and $k$ are relatively small and thus no over or underflow is expected for any of the functions.
For the last two rows, where $\lambda$ is relatively large, we find that the first and third function give the same result, while the result of the second function deviates.
As potential under and overflow errors should not be exactly the same for the first and third function, this leads us to think that
these functions actually give the correct answer (at least up to these 6 significant digits), while the second function then gives an incorrect result.
The problem for the second function is here in the factor $e^{-\lambda}$, which for $\lambda=100$ is of order $10^{-44}$ and can thus not be stored in a 32 bit float.
The values $\lambda,k=2.6,40$ gives a slightly different result for each of the three functions.
According to WolframAlpha, the result of the second function is actually the most accurate here, after which the result of the third function is a close second.
It is not clear to us why the result of the first function deviates here, but this might have something to do with the fact that $\lambda$ is relatively small while
$k$ is relatively large (we found in class that this value of $k$ already gives problems with the original implementation of the Poisson distribution), or with the fact that
$\lambda=2.6$ is not an integer.

In conclusion, the third function gives overall the best result.